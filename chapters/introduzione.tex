\chapter{Introduzione}
\label{cap:introduzione}

\section{L'azienda}

Descrizione dell'azienda.

\section{Aspettative personali}

Descrizione delle aspettative personali.

\section{Organizzazione del testo}

\begin{description}
    \item[{\hyperref[cap:descrizione-stage]{Il secondo capitolo}}] descrive l'attività di stage definendo il problema da affrontare e i vincoli da rispettare lungo lo sviluppo.
    
    \item[{\hyperref[cap:analisi-requisiti]{Il terzo capitolo}}] approfondisce l'analisi dei requisiti effettuata, elencando i casi d'uso raccolti e definendo i rispettivi requisiti.
    
    \item[{\hyperref[cap:strumenti-utilizzati]{Il quarto capitolo}}] approfondisce gli strumenti e le tecnologie utilizzate nello sviluppo del prodotto commissionato e che si pongono come vincolo di progetto.
    
    \item[{\hyperref[cap:progettazione]{Il quinto capitolo}}] approfondisce la progettazione, caratterizzata dall'architettura del sistema e dai design pattern utilizzati.
    
    \item[{\hyperref[cap:prodotto-finale]{Il sesto capitolo}}] illustra il prodotto finale in tutte le sue componenti.
    
    \item[{\hyperref[cap:conclusioni]{Nel settimo capitolo}}] si possono trovare le conclusioni sul lavoro svolto e formate dai test effettuati e da un resoconto finale che pone a confronto le aspettative iniziali con quelli che sono stati i traguardi raggiunti al termine dell'esperienza di stage.
\end{description}

\noindent Riguardo la stesura del testo, relativamente al documento sono state adottate le seguenti convenzioni tipografiche:
\begin{itemize}
	\item gli acronimi, le abbreviazioni e i termini ambigui o di uso non comune menzionati vengono definiti nel glossario, situato alla fine del presente documento;
	\item per la prima occorrenza dei termini riportati nel glossario viene utilizzata la seguente nomenclatura: \emph{parola}\glsfirstoccur;
	\item i termini in lingua straniera o facenti parti del gergo tecnico sono evidenziati con il carattere \emph{corsivo}.
\end{itemize}
