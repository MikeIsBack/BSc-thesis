\chapter{Introduzione}
\label{cap:introduzione}

\intro{Il seguente capitolo ha la funzione di introdurre l'azienda ospitante presso la quale è stato svolto lo stage. Vengono inoltre elencate quelle che sono le aspettative personali principalmente riguardo la crescita tecnica e professionale.}

\setlength{\parskip}{3ex}

\section{L'azienda}

\begin{figure}[!h]
	\centering
	\includegraphics[width=6cm]{../images/CWBI-logo.png}
	\caption{Logo dell'azienda}
	\label{fig:azienda}
\end{figure}

\setlength{\parskip}{3ex}

\noindent CWBI (logo in {\hyperref[fig:azienda]{figura 1.1}}), acronimo di "Codice Web Banking Innovation", è un’azienda italiana che opera nel mercato dell'Information Communication Technology e supporta i propri clienti nello studio dei modelli di business, nella definizione dei processi organizzativi e nella progettazione e realizzazione di software con un forte orientamento alle nuove tecnologie.

\setlength{\parskip}{3ex}

\noindent Fondata a Padova nell'anno 2013, CWBI ha fidelizzato rapporti di collaborazione con aziende nazionali di primaria importanza attraverso la sua struttura interna costituita da professionisti con skills elevate, che negli anni hanno maturato un forte know-how in diversi settori di business quali: Banking, Media and Publishing, Insurance, Industry.

\setlength{\parskip}{3ex}

\noindent Anni di esperienza permettono all'azienda di affrontare con successo ogni singolo aspetto del ciclo di vita dei progetti nei quali è coinvolta; entusiasmo e visione strategica, accompagnati da un forte orientamento al risultato, sono il motore della sua capacità innovativa.

\setlength{\parskip}{3ex}

\pagebreak

\noindent CWBI offre una vasta gamma di servizi, tra cui:
\begin{itemize}
\item sviluppo di applicazioni e portali web-based;
\item sviluppo di applicazioni mobile;
\item studio di fattibilità e sostenibilità dei modelli di business;
\item analisi e definizione dei processi organizzativi;
\item studi di navigabilità e usabilità;
\item studi di ergonomia del software.
\end{itemize}

\section{Aspettative personali}
L'obiettivo dell'attività di stage, oltre lo sviluppo del progetto commissionato dall'azienda ospitante, è quello di crescere personalmente e professionalmente tramite un'impronta aziendale caratterizzata da competenza tecnica e professionale.

\setlength{\parskip}{3ex}

\noindent Di seguito vengono riportati gli obiettivi personali a titolo professionale e formativo da raggiungere:
\begin{itemize}
\item apprendimento di Java;
\item apprendimento dei framework Spring e Hibernate;
\item apprendimento del framework Bootstrap;
\item apprendimento del sistema di controllo di versione aziendale;
\item apprendimento dei processi aziendali;
\item apprendimento degli strumenti per la gestione di progetto;
\item apprendimento dell’ambiente di sviluppo;
\item studio di fattibilità del progetto e realizzazione dello stesso mediante l'utilizzo delle tecnologie aziendali;
\item capacità di trovare soluzioni alternative da quelle proposte.
\end{itemize}

\pagebreak

\section{Organizzazione del testo}

{\hyperref[cap:descrizione-stage]{Il secondo capitolo}} descrive l'attività di stage definendo il problema da affrontare e i vincoli da rispettare lungo lo sviluppo.
    
\noindent {\hyperref[cap:analisi-requisiti]{Il terzo capitolo}} approfondisce l'analisi dei requisiti effettuata, elencando i casi d'uso raccolti e definendo i rispettivi requisiti.
    
\noindent {\hyperref[cap:strumenti-utilizzati]{Il quarto capitolo}} approfondisce gli strumenti e le tecnologie utilizzate nello sviluppo del prodotto commissionato e che si pongono come vincolo di progetto.
    
\noindent {\hyperref[cap:progettazione]{Il quinto capitolo}} approfondisce la progettazione, caratterizzata dall'approccio MDA e dall'architettura del sistema.
    
\noindent {\hyperref[cap:prodotto-finale]{Il sesto capitolo}} illustra il prodotto finale in tutte le sue componenti.
    
\noindent {\hyperref[cap:conclusioni]{Nel settimo capitolo}} si possono trovare le conclusioni sul lavoro svolto, formate dai test effettuati e da un resoconto finale che pone a confronto le aspettative iniziali con quelli che sono stati i traguardi raggiunti al termine dell'esperienza di stage.

\noindent Riguardo la stesura del testo, relativamente al documento sono state adottate le seguenti convenzioni tipografiche:
\begin{itemize}
	\item gli acronimi, le abbreviazioni e i termini ambigui o di uso non comune menzionati vengono definiti nel glossario, situato alla fine del presente documento;
	\item per la prima occorrenza dei termini riportati nel glossario viene utilizzata la seguente nomenclatura: {\hyperref[cap:glossario]{parola}}\glsfirstoccur;
	\item per la bibliografia e sitografia riguardo un certo argomento viene utilizzata la seguente nomenclatura:  {\hyperref[cap:sitografia]{argomento}}\ap{{[b]}}.
\end{itemize}
