\chapter{Descrizione dello stage}
\label{cap:descrizione-stage}

\intro{Il seguente capitolo ha la funzione di introdurre l'attività di stage definendo quella che è la richiesta aziendale e i vincoli di progetto da rispettare nello sviluppo dell'applicativo commissionato.}

\setlength{\parskip}{3ex}

\section{Il problema}
L'azienda CWBI realizza prodotti software su commissione diretta dei clienti, oltre ad applicativi pensati per raggiungere nuovi acquirenti. L'azienda necessita di una metodologia efficiente per tenere traccia dei contatti con i clienti e delle offerte formulate per uno o più  progetti di interesse. Attualmente la soluzione aziendale si basa su una cartella in rete dove sono contenuti tutti i documenti e i contatti con i clienti negli ultimi 10 anni di attività.

\setlength{\parskip}{3ex}

\noindent Ciò che l'azienda desidera è un applicativo web che consenta di creare, organizzare e tenere traccia di queste informazioni in modo da creare un ambiente di lavoro/tracciamento più efficiente ed efficace. 

\section{Il progetto}
L'obiettivo dello stage riguarda quindi lo sviluppo di un modulo web per la webapp aziendale \textit{CW GEST} preesistente. In particolare ciò che deve essere sviluppato è un sistema di gestione delle campagne a supporto dell'area marketing nel contesto aziendale, ovvero l'ambiente bancario/fintech. 

\setlength{\parskip}{3ex}

\noindent La webapp sarà utilizzata dai dipendenti di CWBI e dovrà consentire di aggiungere, visualizzare e aggiornare la lista completa dei propri clienti, con annessi i progetti richiesti e le offerte formulate per essi. 

\setlength{\parskip}{3ex}

\noindent In particolare, il flusso principale è il seguente:
\begin{enumerate}
\item contatto con il cliente e raccolta delle informazioni;
\item definizione del progetto;
\item creazione dell'offerta;
\item approvazione dell'offerta.
\end{enumerate}

\section{Vincoli del progetto}
\subsection{Vincoli temporali}
Come citato nel regolamento delle attività di stage il tempo limite delle attività si colloca tra le 300 e le 320 ore; motivo per il quale questo risulta essere il tempo limite nel realizzare il prodotto commissionato. È stato quindi redatto un piano di lavoro, tenente conto del limite massimo consentito, che organizza le attività in 8 settimane individualmente composte da 40 ore lavorative.  

\subsection{Vincoli metodologici}
Per il versionamento del codice l'azienda ha imposto l'utilizzo di SVN, un servizio di repository utilizzato dall'azienda. 

\subsection{Vincoli tecnologici}
Al fine di raggiungere gli obiettivi prefissati l'azienda ha imposti i seguenti vincoli tecnologici:
\begin{itemize}
\item utilizzo di Java e dei framework Spring MVC e Hibernate per lo sviluppo lato server;
\item utilizzo del framework Bootstrap, JSTL e della libreria jQuery per lo sviluppo lato client;
\item utilizzo di HTML5, CSS3 e JavaScript come supporto allo sviluppo lato client.
\end{itemize}
Non sono stati imposti limiti sull’utilizzo di eventuali tecnologie aggiuntive.
