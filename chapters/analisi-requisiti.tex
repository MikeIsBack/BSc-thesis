\chapter{Analisi dei requisiti}
\label{cap:analisi-requisiti}

\intro{Il seguente capitolo assume la funzione di illustrare i casi d'uso e i requisiti raccolti nella fase di analisi e riguardanti il progetto commissionato.}

\setlength{\parskip}{3ex}

\section{Casi d'uso}
Per poter capire e studiare a fondo tutte le funzionalità che devono essere messe a disposizione dell'utente che utilizza l'applicativo da sviluppare, sono stati realizzati i relativi diagrammi dei casi d'uso di tipo UML. Tali diagrammi sono risultati fondamentali per individuare correttamente tutti i requisiti del sistema in questione.

\setlength{\parskip}{3ex}

\noindent Ciascun caso d'uso è costituito da:
\begin{itemize}
\item attore primario;
\item precondizione;
\item postcondizione;
\item scenario principale;
\end{itemize}

\setlength{\parskip}{3ex}

\noindent I casi d'uso identificati dalla sigla "UCE" rappresentano un caso d'uso d'errore.

\pagebreak


\section{Definizione e tracciamento dei requisiti}

