\chapter{Strumenti utilizzati}
\label{cap:strumenti-utilizzati}

\intro{Il seguente capitolo ha la funzione di introdurre tutti gli strumenti a supporto delle attività di sviluppo della webapp, tra cui: ambiente di lavoro, framework, strumenti di codifica e le librerie a supporto della codifica.}

\setlength{\parskip}{3ex}

\section{Ambiente di lavoro}
CWBI ha a disposizione circa 30 macchine fisiche con sistema operativo Debian, ciascuna delle quali viene utilizzata come supporto per delle macchine virtuali in ambiente Windows 7. Il lavoro quotidiano viene svolto in ambiente virtuale e non fisico; per la connessione alla propria macchina virtuale viene utilizzato il software \textit{VMware Workstation}. Il vantaggio nell'utilizzare macchine virtuali sta nel fatto che tutti i dipendenti presentano la medesima configurazione dell'ambiente di sviluppo e, essendo il lavoro in azienda molto collaborativo, al presentarsi di un problema o di un dubbio sul "cosa fare" i colleghi possono connettersi alla macchina virtuale dove si è presentato il problema per aiutare e velocizzare lo scambio di idee sul come fare.

\section{Framework}
Nella seguente sezione sono riportati i framework utilizzati nelle attività di sviluppo del progetto commissionato. Tali tecnologie erano già state definite come vincolo progettuale a inizio stage e se ne riporta un'apposita spiegazione.

\pagebreak

\subsection{Spring}

\begin{figure}[!h]
	\centering
	\includegraphics[width=4cm]{../images/Spring-logo.png}
	\caption{Logo di Spring}
\end{figure}

\noindent Spring è un {\hyperref[cap:framework-definition]{framework}}\glsfirstoccur open source nato per lo sviluppo di applicazioni Java enterprise. Una delle sue maggiori peculiarità risiede nell'essere "modulare", il che consente di utilizzare anche solo una parte delle funzionalità che mette a disposizione. Altro punto di forza è la facile integrazione con altri framework esistenti, come: Hibernate, Apache Struts, ecc. \\
Spring offre inoltre un'architettura MVC per lo sviluppo di progetti software.

\subsection{Hibernate}

\begin{figure}[!h]
	\centering
	\includegraphics[width=5cm]{../images/Hibernate-logo.png}
	\caption{Logo di Hibernate}
\end{figure}

\noindent Hibernate è un framework open source che permette di  rendere persistenti i dati dall'ambiente Java al database mappando gli oggetti in apposite tabelle di un database relazionale. \\
Hibernate gestisce inoltre l'accesso e la manipolazione dei dati stessi generando automaticamente le query in linguaggio SQL; questo permette di agevolare il lavoro dello sviluppatore, che non si deve occupare della scrittura manuale delle query di accesso ai dati del database ma solamente di definire gli oggetti che andranno memorizzati e gestiti.

\subsection{Bootstrap}


\subsection{Apache Struts}


\section{Strumenti di codifica}
\subsection{Java}

\subsection{HTML5, CSS3 e JavaScript}


\section{Librerie a supporto della codifica}
\subsection{Apache commons}


\subsection{JSTL}


\subsection{WRO4J}


\section{IDE}
\subsection{Eclipse}
