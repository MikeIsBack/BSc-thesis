\chapter{Conclusioni}
\label{cap:conclusioni}

\intro{Il seguente capitolo ha la funzione di illustrare l'attività di test eseguita sul prodotto finale. Viene inoltre proposto un resoconto complessivo sull'attività di stage.}

\setlength{\parskip}{3ex}

\section{Test e validazione}
L'attività conclusiva del percorso di stage è stata quella di testare e verificare insieme al tutor aziendale il prodotto finale realizzato secondo i requisiti individuati e raccolti nell'attività di analisi.

\setlength{\parskip}{3ex} 

\noindent Si riporta quindi una tabella riassuntiva dei {\hyperref[para:test-definition]{test di sistema}}\glsfirstoccur \; effettuati sul prodotto finale. Ciascun test è caratterizzato da:
\begin{itemize}
\item requisito di riferimento;
\item test;
\item esito del test.
\end{itemize}

\pagebreak

\subsection{Test di sistema}
\begin{table}[!h]
	\centering
	\begin{tblr}{
		colspec={|c|X|c|},
		row{odd}={bg=white},
		row{even}={bg=gray!30},
		row{1}={bg=white,fg=black}
		}
		\hline 
		\textbf{Requisito} & \textbf{Test} & \textbf{Esito} \\
		\hline
		RFO1 &	Si verifica che l’utente possa utilizzare il menu. &	Superato \\
RFO2 &	Si verifica che l’utente possa creare un nuovo progetto. &	Superato \\
RFO3 &	Si verifica che l’utente possa accedere alla tabella dei progetti. &	Superato \\
RFO4 &	Si verifica che l’utente possa accedere alla tabella dei progetti aperti. &	Superato \\
RFO5 &	Si verifica che l’utente possa accedere alla tabella dei clienti. &	Superato \\
RFO6 &	Si verifica che l’utente possa accedere alla tabella delle offerte. &	Superato \\
RFO7 &	Si verifica che l’utente inserisca il codice del progetto. &	Superato \\
RFO8 &	Si verifica che l’utente inserisca il titolo del progetto. &	Superato \\
RFO9 &	Si verifica che l’utente inserisca la descrizione del progetto. &	Superato \\
RFO10 &	Si verifica che l’utente inserisca la data di inizio del progetto. &	Superato \\
RFO11 & 	Si verifica che l’utente selezioni il cliente committente del progetto. &	Superato \\
RFO12 &	Si verifica che l’utente selezioni il responsabile interno del progetto. & Superato \\
RFO13 &	Si verifica che l’utente selezioni lo stato di avanzamento del progetto. &	Superato \\
RFO14 &	Si verifica che l’utente inserisca i giorni di sviluppo previsti. &	Superato \\
RFO15 &	Si verifica che l’utente selezioni la percentuale di avanzamento del progetto. &	Superato \\
RFO16 &	Si verifica che l’utente visualizzi un errore quando lascia vuoto il campo codice del progetto. &	Superato \\
RFO17 &	Si verifica che l’utente visualizzi un errore quando lascia vuoto il campo titolo del progetto. &	Superato \\
RFO18 &	Si verifica che l’utente visualizzi un errore quando lascia vuoto il campo descrizione del progetto. &	Superato \\
RFO19 & Si verifica che l’utente visualizzi un errore quando lascia vuoto il campo data di inizio del progetto. &	Superato \\
RFO20 &	Si verifica che l’utente visualizzi un errore quando lascia vuoto il campo giorni di sviluppo del progetto. &	Superato \\
RFO21 &	Si verifica che l’utente possa ricercare uno o più progetti. & Superato \\
RFO22 & 	Si verifica che l’utente cerchi un progetto per ID. &	Superato \\
RFO23 &	Si verifica che l’utente cerchi un progetto per cliente associato. &	Superato \\
		\hline
	\end{tblr}
\end{table}

\pagebreak

\begin{table}[!h]
	\centering
	\begin{tblr}{
		colspec={|c|X|c|},
		row{odd}={bg=white},
		row{even}={bg=gray!30},
		row{1}={bg=white,fg=black}
		}
		\hline 
		\textbf{Requisito} & \textbf{Descrizione} & \textbf{Esito} \\
		\hline
RFO24 &	Si verifica che l’utente cerchi un progetto per responsabile interno. &	Superato \\
RFO25 &	Si verifica che l’utente cerchi un progetto per codice di progetto. &	Superato \\
RFO26 & Si verifica che l’utente cerchi un progetto per titolo. &	Superato \\ 
RFO27 &	Si verifica che l’utente cerchi un progetto per stato di avanzamento. &	Superato \\
RFO28 &	Si verifica che l’utente cerchi un progetto per data di creazione. &	Superato \\
RFO29 &	Si verifica che l’utente visualizzi la lista dei progetti.&	Superato \\
RFO30 &	Si verifica che l’utente visualizzi il dettaglio di un progetto. &	Superato \\
RFO31 &	Si verifica che l’utente visualizzi l’ID di un progetto. &	Superato \\
RFO32 &	Si verifica che l’utente visualizzi la stringa “codice - titolo” di un progetto. &	Superato \\
RFO33 &	Si verifica che l’utente visualizzi il committente di un progetto.&	Superato \\
RFO34 &	Si verifica che l’utente visualizzi il responsabile interno di un progetto.&	Superato \\
RFO35 &	Si verifica che l’utente visualizzi lo stato di avanzamento di un progetto.&	Superato \\
RFO36 &	Si verifica che l’utente visualizzi la data di inizio di un progetto.&	Superato \\
RFO37 &	Si verifica che l’utente visualizzi i giorni di sviluppo previsti di un progetto.&	Superato \\
RFO38 &	Si verifica che l’utente visualizzi la descrizione del progetto.&	Superato \\
RFO39 &	Si verifica che l’utente visualizzi la percentuale di avanzamento del progetto.&	Superato \\
RFO40 &	Si verifica che l’utente visualizzi e gestisce i clienti del progetto.&	Superato \\
RFO41 &	Si verifica che l’utente modifichi il progetto.&	Superato \\
RFO42 &	Si verifica che l’utente elimini il progetto.&	Superato \\
RFO43 &	Si verifica che l’utente cerchi uno o più clienti. &	Superato \\
RFO44 &	Si verifica che l’utente cerchi un cliente per ID. &	Superato \\
RFO45 &	Si verifica che l’utente cerchi un cliente per codice. &	Superato \\
RFO46 &	Si verifica che l’utente cerchi un cliente per descrizione. &	Superato \\
RFO47 &	Si verifica che l’utente visualizzi la lista dei clienti. &	Superato \\
RFO48 &	Si verifica che l’utente visualizzi il dettaglio di un cliente. &	Superato \\
RFO49 &	Si verifica che l’utente visualizzi l’ID di un cliente. &	Superato \\ 
		\hline
	\end{tblr}
\end{table}

\pagebreak

\begin{table}[!h]
	\centering
	\begin{tblr}{
		colspec={|c|X|c|},
		row{odd}={bg=white},
		row{even}={bg=gray!30},
		row{1}={bg=white,fg=black}
		}
		\hline 
		\textbf{Requisito} & \textbf{Descrizione} & \textbf{Esito} \\
		\hline
RFO50 &	Si verifica che l’utente visualizzi la stringa “codice – descrizione” di un cliente. &	Superato \\
RFO51 &	Si verifica che l’utente visualizzi partita iva del cliente. &	Superato \\
RFO52 &	Si verifica che l’utente visualizzi il codice fiscale del cliente. &	Superato \\
RFO53 &	Si verifica che l’utente visualizzi la lista dei progetti associati al cliente. &	Superato \\
RFO54 &	Si verifica che l’utente visualizzi la lista delle offerte associate al cliente.  &	Superato \\
RFO55 &	Si verifica che l’utente crei una nuova offerta. &	Superato \\
RFO56 &	Si verifica che l’utente visualizzi la lista delle offerte. &	Superato \\
RFO57 &	Si verifica che l’utente inserisca il titolo dell’offerta. &	Superato \\
RFO58 &	Si verifica che l’utente inserisca la descrizione dell’offerta. &	Superato \\
RFO59 &	Si verifica che l’utente selezioni il progetto coinvolto nell’offerta. &	Superato \\
RFO60 &	Si verifica che l’utente selezioni lo stato attuale dell’offerta. &	Superato \\
RFO61 &	Si verifica che l’utente inserisca l’imponibile 1 dell’offerta. &	Superato \\
RFO62 &	Si verifica che l’utente inserisca l’imponibile 2 dell’offerta. &	Superato \\
RFO63 &	Si verifica che l’utente inserisca l’imponibile 3 dell’offerta. &	Superato \\
RFO64 &	Si verifica che l’utente visualizzi un errore quando lascia vuoto il campo titolo dell’offerta. &	Superato \\
RFO65 &	Si verifica che l’utente visualizzi un errore quando lascia vuoto il campo descrizione dell’offerta. &	Superato \\
RFO66 &	Si verifica che l’utente visualizzi un errore quando lascia vuoto il campo imponibile 1 dell’offerta. &	Superato \\
RFO67 &	Si verifica che l’utente visualizzi un errore quando il tipo di valore inserito del campo imponibile 1 non è numerico. &	Superato \\
RFO68 &	Si verifica che l’utente visualizzi un errore quando il tipo di valore inserito del campo imponibile 2 non è numerico. &	Superato \\
RFO69 &	Si verifica che l’utente visualizzi un errore quando il tipo di valore inserito del campo imponibile 3 non è numerico. &	Superato \\
RFO70 &	Si verifica che l’utente cerchi una o più offerte. &	Superato \\
RFO71 &	Si verifica che l’utente cerchi un’offerta per ID. &	Superato \\
RFO72 &	Si verifica che l’utente cerchi un’offerta per cliente associato. &	Superato \\
RFO73 &	Si verifica che l’utente cerchi un’offerta per stato attuale. &	Superato \\
RFO74 &	Si verifica che l’utente cerchi un’offerta per data di stipula. &	Superato \\
RFO75 &	Si verifica che l’utente visualizzi il dettaglio di un’offerta. &	Superato \\
		\hline
	\end{tblr}
\end{table}

\pagebreak

\begin{table}[!h]
	\centering
	\begin{tblr}{
		colspec={|c|X|c|},
		row{odd}={bg=white},
		row{even}={bg=gray!30},
		row{1}={bg=white,fg=black}
		}
		\hline 
		\textbf{Requisito} & \textbf{Descrizione} & \textbf{Esito} \\
		\hline
RFO76 &	Si verifica che l’utente visualizzi l’ID di un’offerta. &	Superato \\
RFO77 &	Si verifica che l’utente visualizzi la data di stipula di un’offerta. &	Superato \\
RFO78 &	Si verifica che l’utente visualizzi il cliente coinvolto di un’offerta. &	Superato \\
RFO79 &	Si verifica che l’utente visualizzi lo stato di un’offerta. &	Superato \\
RFO80 &	Si verifica che l’utente visualizzi il titolo di un’offerta. &	Superato \\
RFO81 &	Si verifica che l’utente visualizzi il progetto di un’offerta. &	Superato \\
RFO82 &	Si verifica che l’utente visualizzi la descrizione dell’offerta. &	Superato \\
RFO83 &	Si verifica che l’utente visualizzi l’imponibile 1 dell’offerta. &	Superato \\
RFO84 &	Si verifica che l’utente visualizzi l’imponibile 2 dell’offerta. &	Superato \\
RFO85 &	Si verifica che l’utente visualizzi l’imponibile 3 dell’offerta. &	Superato \\
RFO86 &	Si verifica che l’utente modifichi l’offerta. &	Superato \\
RFO87 &	Si verifica che l’utente elimini l’offerta. &	Superato \\
RVO1 &	Si verifica che l'applicativo lato back-end sia realizzato in Java. &	Superato \\
RVO2 &	Si verifica che l'applicativo lato back-end sia realizzato mediante il framework Spring. &	Superato \\
RVO3 &	Si verifica che l'applicativo lato back-end sia realizzato mediante il framework Hibernate. & 	Superato \\
RVO4 &	Si verifica che l'applicativo lato fron-end sia realizzato tramite HTML5, CSS3 e JavaScript. &	Superato \\
RVO5 &	Si verifica che l'applicativo lato front-end sia realizzato mediante il framework Bootstrap. &	Superato \\
RVO6 &	Si verifica che l'applicativo sia funzionante in tutte le sue componenti. &	Superato \\
		\hline
	\end{tblr}
	\setlength{\parskip}{2ex}
	\caption{Test di sistema}
\end{table}

\noindent Tutti i requisiti richiesti, obbligatori e di vincolo, sono stati soddisfatti con un riscontro positivo anche da parte del tutor aziendale. Il prodotto risulta completo di tutte le richieste iniziali; questo tuttavia non esclude una sua evoluzione futura.

\pagebreak

\section{Resoconto dello stage}

\subsection{Lavoro svolto}
Il risultato dell'attività di stage consiste in un prodotto che consente ai dipendenti dell'area commerciale di CWBI di tenere traccia delle offerte formulate ai clienti per i diversi progetti a disposizione dell'azienda stessa e non. L'utente può inserire, visualizzare o modificare i progetti e le offerte formulate per i clienti stessi. È inoltre possibile visualizzare la lista dei clienti con i progetti e le offerte a loro associate. 

\setlength{\parskip}{3ex}

\noindent Il prodotto realizzato è accessibile dalla webapp preesistente CW GEST e ogni operazione messa a disposizione dall'applicativo, che sia visualizzare, aggiungere, modificare o eliminare un elemento, è disponibile solamente agli utenti che possiedono le autorizzazioni necessarie a svolgere queste funzionalità. Le autorizzazioni vengono gestite da un ulteriore modulo preesistente che si occupa di assegnare a ciascun dipendente di CWBI delle etichette in base a ciò che può fare all'interno di CW GEST.

\subsection{Difficoltà incontrate}
Le difficoltà incontrate nello sviluppo del nuovo modulo per CW GEST sono principalmente relative all'utilizzo di tecnologie che, personalmente, non erano mai state affrontate, quali: il linguaggio Java e i framework Spring, Hibernate, Bootstrap e Apache Struts. Al fine di familiarizzare con queste ultime si è deciso di dedicare una porzione iniziale dello stage ad una formazione caratterizzata da ricerche personali, integrate poi con una spiegazione più approfondita del tutor aziendale, e alcune esercitazioni pratiche sugli argomenti affrontati. La formazione mi ha quindi permesso di capire come queste tecnologie lavorano e quali son i principi cardine alla loro base.  

\setlength{\parskip}{3ex}

\noindent Una volta entrato in possesso delle conoscenze necessarie per svolgere il prodotto commissionato non si sono presentate ulteriori difficoltà anche grazie all'utilizzo della documentazione che ha contribuito positivamente nello sviluppo del prodotto. 

\subsection{Metodologie di lavoro aziendali}
Un punto di forza dell'azienda CWBI sta nell'analizzare esattamente ciò che il cliente desidera e fornire la soluzione migliore. Tutto ciò avviene grazie ad un'attenta e accurata attività di analisi del problema e dell'individuazione del requisito a partire dalla richiesta del cliente. Principio cardine del lavoro aziendale è l'approccio MDA che concentra la maggior parte del lavoro nella definizione del modello. Un modello solido consente infatti di poter codificare il prodotto in un tempo adeguato, senza preoccuparsi troppo dell'architettura sottostante dato che questa viene messa a disposizione dal framework Apache Struts. 

\setlength{\parskip}{3ex}

\noindent Per il versionamento del codice l'azienda utilizza il software TortoiseSVN, uno strumento integrato nell'IDE Eclipse. Questo tool consente di sincronizzare i cambiamenti apportati, risolvere eventuali conflitti e rilasciare nuovo codice direttamente dall'IDE di lavoro. Il vantaggio di tutto ciò risiede nell'avere un unico strumento di lavoro, Eclipse, che si occupa di tutto il processo di sviluppo.

\subsection{Retrospettiva personale}
L'attività di stage mi ha permesso di entrare a far parte di un nuovo lato della programmazione, ovvero lo sviluppo di applicazioni Java EE. Le metodologie di approccio aziendali volte all'affrontare nuovi progetti e attività sono state stimolanti e rappresentano un metodo di lavoro solido e duraturo che hanno sicuramente portato ad una mia crescita professionale. 

\setlength{\parskip}{3ex}

\noindent Lo stage mi ha permesso di conoscere e apprendere nuove tecnologie molto utilizzate nella programmazione, tra cui: Spring, Hibernate, Bootstrap e Apache Struts. Questa conoscenza appresa ha arricchito il mio bagaglio informatico.

\setlength{\parskip}{3ex}

\noindent Ho avuto inoltre la possibilià di lavorare con un team di esperti e appassionati del settore che mi hanno trasmesso la loro conoscenza e la passione con la quale operano ogni giorno nel mondo dello sviluppo software.