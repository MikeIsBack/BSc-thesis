\chapter{Conclusioni}
\label{cap:conclusioni}

\intro{Il seguente capitolo ha la funzione di illustrare l'attività di test eseguita sul prodotto finale. Viene proposto anche un resoconto complessivo sull'attività di stage.}

\setlength{\parskip}{3ex}

\section{Test e validazione}
L'attività conclusiva del percorso di stage è stata quella di testare e verificare insieme al tutor aziendale il prodotto finale composto dai requisiti individuati nell'attività di analisi.

\setlength{\parskip}{3ex} 

\noindent Si riporta quindi una tabella riassuntiva dei {\hyperref[para:test-definition]{test di sistema}}\glsfirstoccur \; effettuati sul prodotto finale. Ciascun test è caratterizzato dal requisito di riferimento, una descrizione e l'esito.

\pagebreak

\subsection{Test di sistema}
\begin{table}[!h]
	\centering
	\begin{tblr}{
		colspec={|c|X|c|},
		row{odd}={bg=white},
		row{even}={bg=gray!30},
		row{1}={bg=white,fg=black}
		}
		\hline 
		\textbf{Requisito} & \textbf{Descrizione} & \textbf{Esito} \\
		\hline
		RFO1 &	L’utente deve poter utilizzare il menu. &	Superato \\
RFO2 &	L’utente deve poter creare un nuovo progetto. &	Superato \\
RFO3 &	L’utente deve poter accedere alla tabella dei progetti. &	Superato \\
RFO4 &	L’utente deve poter accedere alla tabella dei progetti aperti. &	Superato \\
RFO5 &	L’utente deve poter accedere alla tabella dei clienti. &	Superato \\
RFO6 &	L’utente deve poter accedere alla tabella delle offerte. &	Superato \\
RFO7 &	L’utente inserisce il codice del progetto. &	Superato \\
RFO8 &	L’utente inserisce il titolo del progetto. &	Superato \\
RFO9 &	L’utente inserisce la descrizione del progetto. &	Superato \\
RFO10 &	L’utente inserisce la data di inizio del progetto. &	Superato \\
RFO11 & 	L’utente seleziona il cliente committente del progetto. &	Superato \\
RFO12 &	L’utente seleziona il responsabile interno del progetto. & Superato \\
RFO13 &	L’utente seleziona lo stato di avanzamento del progetto. &	Superato \\
RFO14 &	L’utente inserisce i giorni di sviluppo previsti. &	Superato \\
RFO15 &	L’utente seleziona la percentuale di avanzamento del progetto. &	Superato \\
RFO16 &	L’utente visualizza un errore quando lascia vuoto il campo codice del progetto. &	Superato \\
RFO17 &	L’utente visualizza un errore quando lascia vuoto il campo titolo del progetto. &	Superato \\
RFO18 &	L’utente visualizza un errore quando lascia vuoto il campo descrizione del progetto. &	Superato \\
RFO19 & L’utente visualizza un errore quando lascia vuoto il campo data di inizio del progetto. &	Superato \\
RFO20 &	L’utente visualizza un errore quando lascia vuoto il campo giorni di sviluppo del progetto. &	Superato \\
RFO21 &	L’utente può ricercare uno o più progetti. & Superato \\
RFO22 & 	L’utente ricerca un progetto per ID. &	Superato \\
RFO23 &	L’utente ricerca un progetto per cliente associato. &	Superato \\
RFO24 &	L’utente ricerca un progetto per responsabile interno. &	Superato \\
RFO25 &	L’utente ricerca un progetto per codice di progetto. &	Superato \\
RFO26 & L’utente ricerca un progetto per titolo. &	Superato \\ 
		\hline
	\end{tblr}
\end{table}

\pagebreak

\begin{table}[!h]
	\centering
	\begin{tblr}{
		colspec={|c|X|c|},
		row{odd}={bg=white},
		row{even}={bg=gray!30},
		row{1}={bg=white,fg=black}
		}
		\hline 
		\textbf{Requisito} & \textbf{Descrizione} & \textbf{Esito} \\
		\hline
RFO27 &	L’utente ricerca un progetto per stato di avanzamento. &	Superato \\
RFO28 &	L’utente ricerca un progetto per data di creazione. &	Superato \\
RFO29 &	L’utente visualizza la lista dei progetti.&	Superato \\
RFO30 &	L’utente visualizza il dettaglio di un progetto. &	Superato \\
RFO31 &	L’utente visualizza l’ID di un progetto. &	Superato \\
RFO32 &	L’utente visualizza la stringa “codice - titolo” di un progetto. &	Superato \\
RFO33 &	L’utente visualizza il committente di un progetto.&	Superato \\
RFO34 &	L’utente visualizza il responsabile interno di un progetto.&	Superato \\
RFO35 &	L’utente visualizza lo stato di avanzamento di un progetto.&	Superato \\
RFO36 &	L’utente visualizza la data di inizio di un progetto.&	Superato \\
RFO37 &	L’utente visualizza i giorni di sviluppo previsti di un progetto.&	Superato \\
RFO38 &	L’utente visualizza la descrizione del progetto.&	Superato \\
RFO39 &	L’utente visualizza la percentuale di avanzamento del progetto.&	Superato \\
RFO40 &	L’utente visualizza e gestisce i clienti del progetto.&	Superato \\
RFO41 &	L’utente modifica il progetto.&	Superato \\
RFO42 &	L’utente elimina il progetto.&	Superato \\
RFO43 &	L’utente ricerca uno o più clienti. &	Superato \\
RFO44 &	L’utente ricerca un cliente per ID. &	Superato \\
RFO45 &	L’utente ricerca un cliente per codice. &	Superato \\
RFO46 &	L’utente ricerca un cliente per descrizione. &	Superato \\
RFO47 &	L’utente visualizza la lista dei clienti. &	Superato \\
RFO48 &	L’utente visualizza il dettaglio di un cliente. &	Superato \\
RFO49 &	L’utente visualizza l’ID di un cliente. &	Superato \\
RFO50 &	L’utente visualizza la stringa “codice – descrizione” di un cliente. &	Superato \\
RFO51 &	L’utente visualizza partita iva del cliente. &	Superato \\
RFO52 &	L’utente visualizza il codice fiscale del cliente. &	Superato \\
RFO53 &	L’utente visualizza la lista dei progetti associati al cliente. &	Superato \\ 
		\hline
	\end{tblr}
\end{table}

\pagebreak

\begin{table}[!h]
	\centering
	\begin{tblr}{
		colspec={|c|X|c|},
		row{odd}={bg=white},
		row{even}={bg=gray!30},
		row{1}={bg=white,fg=black}
		}
		\hline 
		\textbf{Requisito} & \textbf{Descrizione} & \textbf{Esito} \\
		\hline
RFO54 &	L’utente visualizza la lista delle offerte associate al cliente.  &	Superato \\
RFO55 &	L’utente crea una nuova offerta. &	Superato \\
RFO56 &	L’utente visualizza la lista delle offerte. &	Superato \\
RFO57 &	L’utente inserisce il titolo dell’offerta. &	Superato \\
RFO58 &	L’utente inserisce la descrizione dell’offerta. &	Superato \\
RFO59 &	L’utente seleziona il progetto coinvolto nell’offerta. &	Superato \\
RFO60 &	L’utente seleziona lo stato attuale dell’offerta. &	Superato \\
RFO61 &	L’utente inserisce l’imponibile 1 dell’offerta. &	Superato \\
RFO62 &	L’utente inserisce l’imponibile 2 dell’offerta. &	Superato \\
RFO63 &	L’utente inserisce l’imponibile 3 dell’offerta. &	Superato \\
RFO64 &	L’utente visualizza un errore quando lascia vuoto il campo titolo dell’offerta. &	Superato \\
RFO65 &	L’utente visualizza un errore quando lascia vuoto il campo descrizione dell’offerta. &	Superato \\
RFO66 &	L’utente visualizza un errore quando lascia vuoto il campo imponibile 1 dell’offerta. &	Superato \\
RFO67 &	L’utente visualizza un errore quando il tipo di valore inserito del campo imponibile 1 non è numerico. &	Superato \\
RFO68 &	L’utente visualizza un errore quando il tipo di valore inserito del campo imponibile 2 non è numerico. &	Superato \\
RFO69 &	L’utente visualizza un errore quando il tipo di valore inserito del campo imponibile 3 non è numerico. &	Superato \\
RFO70 &	L’utente ricerca una o più offerte. &	Superato \\
RFO71 &	L’utente ricerca un’offerta per ID. &	Superato \\
RFO72 &	L’utente ricerca un’offerta per cliente associato. &	Superato \\
RFO73 &	L’utente ricerca un’offerta per stato attuale. &	Superato \\
RFO74 &	L’utente ricerca un’offerta per data di stipula. &	Superato \\
RFO75 &	L’utente visualizza il dettaglio di un’offerta. &	Superato \\
RFO76 &	L’utente visualizza l’ID di un’offerta. &	Superato \\
RFO77 &	L’utente visualizza la data di stipula di un’offerta. &	Superato \\
RFO78 &	L’utente visualizza il cliente coinvolto di un’offerta. &	Superato \\
RFO79 &	L’utente visualizza lo stato di un’offerta. &	Superato \\
		\hline
	\end{tblr}
\end{table}

\pagebreak

\begin{table}[!h]
	\centering
	\begin{tblr}{
		colspec={|c|X|c|},
		row{odd}={bg=white},
		row{even}={bg=gray!30},
		row{1}={bg=white,fg=black}
		}
		\hline 
		\textbf{Requisito} & \textbf{Descrizione} & \textbf{Esito} \\
		\hline
RFO80 &	L’utente visualizza il titolo di un’offerta. &	Superato \\
RFO81 &	L’utente visualizza il progetto di un’offerta. &	Superato \\
RFO82 &	L’utente visualizza la descrizione dell’offerta. &	Superato \\
RFO83 &	L’utente visualizza l’imponibile 1 dell’offerta. &	Superato \\
RFO84 &	L’utente visualizza l’imponibile 2 dell’offerta. &	Superato \\
RFO85 &	L’utente visualizza l’imponibile 3 dell’offerta. &	Superato \\
RFO86 &	L’utente modifica l’offerta. &	Superato \\
RFO87 &	L’utente elimina l’offerta. &	Superato \\
RVO1 &	L'applicativo lato back-end è realizzato in Java. &	Superato \\
RVO2 &	L'applicativo lato back-end è realizzato mediante il framework Spring. &	Superato \\
RVO3 &	L'applicativo lato back-end è realizzato mediante il framework Hibernate. & 	Superato \\
RVO4 &	L'applicativo lato fron-end è realizzato tramite HTML5, CSS3 e JavaScript. &	Superato \\
RVO5 &	L'applicativo lato front-end è realizzato mediante il framework Bootstrap. &	Superato \\
RVO6 &	L'applicativo deve essere funzionante in tutte le sue componenti. &	Superato \\
		\hline
	\end{tblr}
	\setlength{\parskip}{2ex}
	\caption{Test di sistema}
\end{table}

\noindent Tutti i requisiti richiesti, obbligatori e di vincolo, sono stati soddisfatti con un riscontro positivo anche da parte del tutor aziendale. 

\pagebreak

\section{Resoconto dello stage}
L'attività di stage mi ha permesso di entrare a far parte di un nuovo lato della programmazione, ovvero lo sviluppo di applicazione Java EE. Le metodologie di approccio aziendali volte all'affrontare nuovi progetti e attività sono state stimolanti e rappresentano un metodo di lavoro solido e duraturo che hanno sicuramente portato ad una mia crescita professionale. 

\setlength{\parskip}{3ex}

\noindent Lo stage mi ha permesso di conoscere e apprendere nuove tecnologie molto utilizzate nella programmazione, tra cui: Spring, Hibernate, Bootstrap e Struts. Questa conoscenza appresa ha arricchito il mio bagaglio informatico.

\setlength{\parskip}{3ex}

\noindent Ho avuto inoltre la possibilià di lavorare con un team di esperti e appassionati del settore che mi hanno trasmesso la loro conoscenza e la passione e con la quale operano ogni giorno nel mondo dello sviluppo software.