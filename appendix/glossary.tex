\cleardoublepage
\chapter{Glossario}

\nocite{*}

\paragraph*{UML (Unified Modeling Language)} Linguaggio che permette, tramite l'utilizzo di modelli visuali, di analizzare, descrivere, specificare e documentare un sistema software anche complesso.
\label{para:uml-definition}

\paragraph*{Requisito} Descrizione dei servizi che un sistema software deve fornire, insieme ai vincoli da rispettare sia in fase di sviluppo che durante la fase di operatività del software.
\label{para:requisito-definition}

\paragraph*{Framework} Architettura logica di supporto sulla quale un software può essere progettato e realizzato, spesso facilitandone lo sviluppo da parte del programmatore.
\label{para:framework-definition}

\paragraph*{IDE (Integrated Development Environment)} Software che offre agli sviluppatori un ambiente per lo sviluppo, il test e il debug di un'applicazione.
\label{para:ide-definition}

\paragraph*{Test di sistema} Una tipologia di test del software che viene sempre condotto su un intero sistema. Verifica se il sistema è conforme ai suoi requisiti, qualunque essi siano.
\label{para:test-definition}