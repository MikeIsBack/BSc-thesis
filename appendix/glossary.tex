\cleardoublepage
\chapter{Glossario}

\nocite{*}

\section*{UML (Unified Modeling Language)}
\label{sec:uml-definition}
Linguaggio che permette, tramite l'utilizzo di modelli visuali, di analizzare, descrivere, specificare e documentare un sistema software anche complesso.

\section*{Requisito}
\label{sec:requisito-definition}
Descrizione dei servizi che un sistema software deve fornire, insieme ai vincoli da rispettare sia in fase di sviluppo che durante la fase di operatività del software.

\section*{Framework}
\label{sec:framework-definition}
Architettura logica di supporto sulla quale un software può essere progettato e realizzato, spesso facilitandone lo sviluppo da parte del programmatore.

\section*{IDE (Integrated Development Environment)}
\label{sec:ide-definition}
Software che offre agli sviluppatori un ambiente per lo sviluppo, il test e il debug di un'applicazione.

\section*{Test di sistema}
\label{sec:test-definition}
Una tipologia di test del software che viene sempre condotto su un intero sistema. Verifica se il sistema è conforme ai suoi requisiti, qualunque essi siano.